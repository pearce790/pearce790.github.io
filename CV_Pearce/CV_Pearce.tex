%%%%%%%%%%%%%%%%%%%%%%%%%%%%%%%%%%%%%%%%%
% Medium Length Professional CV
% LaTeX Template
% Version 2.0 (8/5/13)
%
% This template has been downloaded from:
% http://www.LaTeXTemplates.com
%
% Original author:
% Rishi Shah 
%
% Important note:
% This template requires the resume.cls file to be in the same directory as the
% .tex file. The resume.cls file provides the resume style used for structuring the
% document.
%
%%%%%%%%%%%%%%%%%%%%%%%%%%%%%%%%%%%%%%%%%

%----------------------------------------------------------------------------------------
%	PACKAGES AND OTHER DOCUMENT CONFIGURATIONS
%----------------------------------------------------------------------------------------

\documentclass{resume} % Use the custom resume.cls style

\usepackage[left=0.75in,top=0.6in,right=0.75in,bottom=0.6in]{geometry}
\usepackage{scrextend}
\usepackage{url}
\newcommand{\tab}[1]{\hspace{.2667\textwidth}\rlap{#1}}
\newcommand{\itab}[1]{\hspace{0em}\rlap{#1}}
\name{Michael Pearce} % Your name
\address{University of Washington, Department of Statistics}
\address{Padelford Hall, Box 354322} % Your address
%\address{123 Pleasant Lane \\ City, State 12345} % Your secondary addess (optional)
\address{mpp790@uw.edu} % Your phone number and email

\begin{document}

%----------------------------------------------------------------------------------------
%	EDUCATION SECTION
%----------------------------------------------------------------------------------------

\begin{rSection}{Education}

{\bf University of Washington}, Seattle, WA \hfill {2018 - Present} 
\\ Ph.D. (anticipated) in statistics
\\ Advised by Elena A. Erosheva
\\
\\{\bf St. Olaf College}, Northfield, MN \hfill {2013 - 2017} 
\\ B.A. in mathematics; concentration in statistics
\\ Graduated {\it summa cum laude}
\end{rSection}

\begin{rSection}{Research Experience}
{\bf Research Assistant} \hfill {2020 - Present}
\\ \textit{University of Washington - Department of Statistics}
\\ Supervisor: Elena A. Erosheva
\\ Topic: Unified modeling of rankings and ratings with application to peer review.


{\bf Research Assistant} \hfill {2020 - 2021}
\\ \textit{University of Washington - Department of Statistics}
\\ Supervisor: Adrian E. Raftery
\\ Topic: Bayesian forecasting of the maximum human lifespan to 2100.

{\bf Statistical Fellow} \hfill {2016 - 2017}
\\ \textit{St. Olaf College - Center for Undergraduate Research}
\\ Supervisors: Rodrigo Sanchez-Gonzalez and Matthew Richey
\\ Topic: Increasing spatial resolution in molecular tagging velocimetry images via unsupervised learning.
\end{rSection}

\begin{rSection}{Scholarly Publications}

{\bf Pearce, M.} and Erosheva, E.A. ``Modeling preferences: A Bayesian mixture of finite mixtures for rankings and ratings" \textit{arXiv preprint arXiv:2301.09755} (2023).

Gallo, S.A., {\bf Pearce, M.}, Lee, C.J., and Erosheva, E.A. ``A new approach to peer review assessments: Score, then rank." \textit{Research Integrity and Peer Review preprint: DOI: 10.21203/rs.3.rs-2198949} (2022).

{\bf Pearce, M.} and Erosheva, E.A. ``On the validity of bootstrap uncertainty estimates in the Mallows-Binomial model." \textit{arXiv preprint arXiv:2206.12365} (2022).

{\bf Pearce, M.} and Erosheva, E.A. ``A unified statistical learning model for rankings and scores with application to grant panel review." \textit{Journal of Machine Learning Research} 23.210 (2022): 1--33.

{\bf Pearce, M.} and Raftery, A.E. ``Probabilistic forecasting of maximum human lifespan by 2100 using Bayesian population projections." {\em Demographic Research} 44.52 (2021): 1271--1294.

{\bf Pearce, M.}$^*$, Sparrow, Z.$^*$, Mabote, T. R., and Sanchez-Gonzalez, R. ``stoBEST: An efficient methodology for increased spatial resolution in two-component molecular tagging velocimetry." {\em Measurement Science and Technology} 32.3 (2020): 035302

{\em $^*$indicates authors contributed equally.}
\end{rSection}
\newpage

\begin{rSection}{Other Publications}

{\bf Pearce, M.} and Raftery, A.E. ``Will this be a record-breaking century for human longevity?" {\em Significance} (2021).

{\bf Pearce, M.} and Raftery, A.E. ``The maximum human life span will likely increase this century, but not by more than a decade" {\em The Conversation} (2021).

\end{rSection}

\begin{rSection}{Software}

{\bf rankrate: Statistical Tools for Preference Learning with Rankings and Ratings.} R package available on CRAN.

\end{rSection}

\begin{rSection}{Selected Media Coverage}

\small{

{\bf BBC News (Brazil)} ``Por que cada vez mais pessoas est\~{a}o vivendo at\'{e} os 100 anos?" (July 11, 2022)

{\bf Stats and Stories (Podcast)} ``The Age of the Supercentenarian" (April 29, 2022) \url{https://statsandstories.net/health1/the-age-of-the-supercentenarian}.

{\bf Washington Post} ``Want to add healthy years to your life? Here’s what new longevity research says." (Oct. 11, 2021)

{\bf Southern Weekly (China)} ``What is the limit of human life span?" (Sept. 16, 2021)

{\bf CNBC} ``Researchers say the probability of living past 110 is on the rise — here’s what you can do to get there" (July 17, 2021)

{\bf Elemental (Medium)} ``How Long Can Humans Really Live?" (July 15, 2021)

{\bf Gulf News} ``Surviving up to 150: How long can a person live?" (July 12, 2021

{\bf Indian Express} ``Can a person live to age 124, 135 or 150? Some optimism, some caveats" (July 6, 2021)

{\bf The South African} ``Rise of the supercentenarians: Today’s kids could live for 130 years" (July 4, 2021)

{\bf UW News} ``How long can a person live? The 21st century may see a record-breaker" (July 1, 2021)
}

\end{rSection}

\begin{rSection}{Teaching and Mentorship}

{\bf Instructor of Record}
\\ {\it University of Washington}
\\ Introduction to R for Social Scientists (CSSS 508) \hfill {Autumn 2022}

{\bf Teaching Assistant}
\\ {\it University of Washington}
\\ Applied Statistics Capstone (STAT 528) \hfill Winter 2021, Winter 2022
\\ Multivariate Data Analysis for the Social Sciences (CSSS 589) \hfill Autumn 2021
\\ Statistics and Philosophy of Voting (STAT 498 / CS&SS 594) \hfill Autumn 2020
\\ Elements of Statistical Methods (STAT 311) \hfill {Autumn 2018, Winter 2019}
\\ Introduction to Probability and Mathematical Statistics III (STAT 342) \hfill {Spring 2019}
\\ Statistical Reasoning (STAT 220) \hfill {Autumn 2019}

{\bf Directed Reading Program Mentor}
\\ {\it University of Washington}
\\ ``Social Choice Analysis of Peer Review Data" \hfill {Spring 2022}
\\ ``Voting, Ranking, and Preference Modeling" \hfill {Autumn 2021}
\\ ``Nonlinear Regression" \hfill {Winter 2020, Winter 2021, Spring 2021}
\\ ``History and Practice of Data Communication" \hfill Autumn 2020

{\bf Washington eXperimental Mathematics Lab (WXML) Mentor}
\\ {\it University of Washington}
\\ ``Improving Panel Consensus Tool (ImPaCT)" \hfill {Autumn 2021}


{\bf Supplemental Instruction Leader}
\\ {\it St. Olaf College}
\\ Calculus II (MATH 126) \hfill {Spring 2017}
\\ Modern Computational Mathematics (MATH 242) \hfill {Spring 2017}

{\bf Student Mentor} 
\\ {\it TRiO Upward Bound}
\\ Mentor for high school students from underrepresented communities in \hfill {2013 - 2015}
\\the Minneapolis and St. Paul public school systems.

\end{rSection}

\begin{rSection}{Professional Experience}

\begin{rSubsection}{Boeing Research and Technology}{2019 - 2020}{Applied Statistics Intern}{}

Performed research involving nonparametric statistics, design of experiments, and aircraft COVID-19 modeling. Formulated, developed, and tested web-based statistical tools for company engineers. Consulted across the company, including end-to-end analysis and communication of findings.
\end{rSubsection}

\begin{rSubsection}{Deloitte LLC}{2017 - 2018}{Analytics Consultant}{}

Verified the accuracy and completeness of complex statistical models for a financial client to ensure compliance with regulatory stress-testing. Analyzed anti-money laundering policies and practices for a global bank.
\end{rSubsection}

\end{rSection}

\begin{rSection}{Conference Participation}

\textbf{NeurIPS}, New Orleans, LA \hfill {December 2022}\\
``A Unified Statistical Learning Model for Rankings and Scores with Application to Grant Panel Review" (Journal-to-Conference Track Poster Session)

\textbf{Joint Statistical Meetings}, Washington, D.C. \hfill {August 2022}\\
``Using ranking data for decision-making" (topic-contributed paper session; organizer and chair)\\
``Fast Bayesian estimation for ranking models" (speed session)

\textbf{ISBA World Meeting}, Montreal, Canada \hfill {June 2022}\\
``Joint Bayesian inference for rankings and ratings under heterogeneous preferences" (poster session)

\textbf{Working Group on Model-Based Clustering}, Athens, Greece (virtual)  \hfill {October 2021}\\
``Unified latent class modeling of scores and rankings applied to grant panel review" (poster session)

\textbf{Joint Statistical Meetings}, Seattle, WA (virtual) \hfill {August 2021}\\
``Unified latent class modeling of score and rank data applied to grant panel review" (speed session)

\textbf{International Conference on Machine Learning} (virtual) \hfill {July 2021}\\
Workflow Chair (ranked data processing)

\textbf{MAA MathFest}, Chicago, IL\hfill {July 2017}\\
``A new method for computational analysis of high-speed gas flows" (Pi Mu Epsilon student paper session)

\textbf{National Conference on Undergraduate Research}, Memphis, TN \hfill {April 2017}\\
``Analysis of high-speed gaseous flows using molecular tagging velocimetry and the Hough transform" (poster presentation)
\end{rSection}

\newpage
\begin{rSection}{Honors and Awards}

\textbf{Scholar Award} {\it (NeurIPS)} \hfill {2022}

\textbf{Dorothy M. Gilford Award} {\it (University of Washington)} \hfill {2021}
\\ {\it ``For outstanding performance by a graduate teaching assistant during the prior year."}

\textbf{Phi Beta Kappa} {\it (St. Olaf College)}\hfill {2017}
\\ Liberal arts honor society

\textbf{Pi Mu Epsilon} {\it (St. Olaf College)} \hfill {2016}
\\ Mathematics honor society; elected treasurer in 2017

\end{rSection}

\begin{rSection}{Reading Group and Lab  Participation}
Statistical and ML Methodology for the Social Sciences Working Group \hfill{2021 - Present}\\
Applied Bayesian and Computational (ABC) Statistics Working Group \hfill {2019 - Present}\\
Statistics Education Reading Group \hfill {2019 - Present}
\end{rSection}

\begin{rSection}{Service at University of Washington}
Pre-Application Review Service (PARS) -- reviewer \hfill{2022}\\
Queer Union for (Bio)statistician Inclusion and Community (QUBIC) -- founder \hfill {2022 - Present}\\
Diversity, Inclusion, Community, and Equity (DICE) Committee -- member \hfill {2020 - Present}\\
Directed Reading Program -- member\hfill{ 2020 - Present}\\
Undergraduate Curriculum Revamp Committee -- member \hfill {2021 - 2022}\\
PhD Admissions Committee -- reviewer \hfill{2020 - 2021}\\
Statistics Education Reading Group -- organizer\hfill{2019 - 2021}


\end{rSection}



\end{document}

\begin{rSection}{Skills}
\begin{tabular}{ @{} >{\bfseries}l @{\hspace{6ex}} l }
Programming \ & R (fluent), Python (proficient) \\
\end{tabular}
\end{rSection}